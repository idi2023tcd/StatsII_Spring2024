\documentclass[12pt,letterpaper]{article}
\usepackage{graphicx,textcomp}
\usepackage{natbib}
\usepackage{setspace}
\usepackage{fullpage}
\usepackage{color}
\usepackage[reqno]{amsmath}
\usepackage{amsthm}
\usepackage{fancyvrb}
\usepackage{amssymb,enumerate}
\usepackage[all]{xy}
\usepackage{endnotes}
\usepackage{lscape}
\newtheorem{com}{Comment}
\usepackage{float}
\usepackage{hyperref}
\newtheorem{lem} {Lemma}
\newtheorem{prop}{Proposition}
\newtheorem{thm}{Theorem}
\newtheorem{defn}{Definition}
\newtheorem{cor}{Corollary}
\newtheorem{obs}{Observation}
\usepackage[compact]{titlesec}
\usepackage{dcolumn}
\usepackage{tikz}
\usetikzlibrary{arrows}
\usepackage{multirow}
\usepackage{xcolor}
\newcolumntype{.}{D{.}{.}{-1}}
\newcolumntype{d}[1]{D{.}{.}{#1}}
\definecolor{light-gray}{gray}{0.65}
\usepackage{url}
\usepackage{listings}
\usepackage{color}

\definecolor{codegreen}{rgb}{0,0.6,0}
\definecolor{codegray}{rgb}{0.5,0.5,0.5}
\definecolor{codepurple}{rgb}{0.58,0,0.82}
\definecolor{backcolour}{rgb}{0.95,0.95,0.92}

\lstdefinestyle{mystyle}{
	backgroundcolor=\color{backcolour},   
	commentstyle=\color{codegreen},
	keywordstyle=\color{magenta},
	numberstyle=\tiny\color{codegray},
	stringstyle=\color{codepurple},
	basicstyle=\footnotesize,
	breakatwhitespace=false,         
	breaklines=true,                 
	captionpos=b,                    
	keepspaces=true,                 
	numbers=left,                    
	numbersep=5pt,                  
	showspaces=false,                
	showstringspaces=false,
	showtabs=false,                  
	tabsize=2
}
\lstset{style=mystyle}
\newcommand{\Sref}[1]{Section~\ref{#1}}
\newtheorem{hyp}{Hypothesis}

\title{Problem Set 4}
\date{Due: April 12, 2024}
\author{Idi Amin Da Silva - Student Number: 23372225, Applied Stats II}



\begin{document}
	\maketitle
	\section*{Instructions}
	\begin{itemize}
	\item Please show your work! You may lose points by simply writing in the answer. If the problem requires you to execute commands in \texttt{R}, please include the code you used to get your answers. Please also include the \texttt{.R} file that contains your code. If you are not sure if work needs to be shown for a particular problem, please ask.
	\item Your homework should be submitted electronically on GitHub in \texttt{.pdf} form.
	\item This problem set is due before 23:59 on Friday April 12, 2024. No late assignments will be accepted.

	\end{itemize}

	\vspace{.25cm}
\section*{Question 1}
\vspace{.25cm}
\noindent We're interested in modeling the historical causes of child mortality. We have data from 26855 children born in Skellefteå, Sweden from 1850 to 1884. Using the "child" dataset in the \texttt{eha} library, fit a Cox Proportional Hazard model using mother's age and infant's gender as covariates. Present and interpret the output.

	\lstinputlisting[language=R,firstline=37,lastline=56]{PS4_Rcode.R}
	
	

	
	\begin{table}[htbp]
		\centering
		\caption{Cox Proportional Hazard Model Summary}
		\label{tab:cox_model_summary}
		\begin{tabular}{lcccccc}
			\toprule
			& \textbf{coef} & \textbf{exp(coef)} & \textbf{se(coef)} & \textbf{z} & \textbf{Pr(\textgreater |z|)} \\
			\midrule
			\textbf{m.age} & 0.007617 & 1.007646 & 0.002128 & 3.580 & 0.000344 \\
			\textbf{sexfemale} & -0.082215 & 0.921074 & 0.026743 & -3.074 & 0.002110 \\
			\bottomrule
		\end{tabular}
		\vspace{0.5cm}
		
		\begin{tabular}{lcccc}
			\toprule
			& \textbf{exp(coef)} & \textbf{exp(-coef)} & \textbf{lower .95} & \textbf{upper .95} \\
			\midrule
			\textbf{m.age} & 1.0076 & 0.9924 & 1.003 & 1.0119 \\
			\textbf{sexfemale} & 0.9211 & 1.0857 & 0.874 & 0.9706 \\
			\bottomrule
		\end{tabular}
		\vspace{0.5cm}
		
		\begin{tabular}{lc}
			\toprule
			& \textbf{Value} \\
			\midrule
			\textbf{Concordance} & 0.519 \\
			\textbf{Likelihood ratio test} & 22.52 \\
			\textbf{Wald test} & 22.52 \\
			\textbf{Score (logrank) test} & 22.53 \\
			\bottomrule
		\end{tabular}
	\end{table}
	With every additional increase in mother's age i.e., increases of one unit of age for mother is associated with a increase of 0.007617 in the expected log of the hazard. There is a 0.082215  decrease in the expected log of the hazard for female babies.
	
	in other words I can interpret this outcome as the following: Interpretation: For each unit increase in mother's age, the hazard (risk) of the event increases by a factor of approximately 1.0076, holding all other variables constant. This effect is statistically significant at the 0.05 level (indicated by ***).
	
	Interpretation: Being female (compared to male) decreases the hazard (risk) of the event by a factor of approximately 0.9211, holding all other variables constant. This effect is statistically significant at the 0.05 level (indicated by **)
\newpage

stargazer(cox_model)

The code above stargazer(cox_model)
Will produce the following table outcome:

% Table created by stargazer v.5.2.3 by Marek Hlavac, Social Policy Institute. E-mail: marek.hlavac at gmail.com
% Date and time: Fri, Apr 12, 2024 - 16:09:54
\begin{table}[!htbp] \centering 
	\caption{} 
	\label{} 
	\begin{tabular}{@{\extracolsep{5pt}}lc} 
		\\[-1.8ex]\hline 
		\hline \\[-1.8ex] 
		& \multicolumn{1}{c}{\textit{Dependent variable:}} \\ 
		\cline{2-2} 
		\\[-1.8ex] & enter \\ 
		\hline \\[-1.8ex] 
		m.age & 0.008$^{***}$ \\ 
		& (0.002) \\ 
		& \\ 
		sexfemale & $-$0.082$^{***}$ \\ 
		& (0.027) \\ 
		& \\ 
		\hline \\[-1.8ex] 
		Observations & 26,574 \\ 
		R$^{2}$ & 0.001 \\ 
		Max. Possible R$^{2}$ & 0.986 \\ 
		Log Likelihood & $-$56,503.480 \\ 
		Wald Test & 22.520$^{***}$ (df = 2) \\ 
		LR Test & 22.518$^{***}$ (df = 2) \\ 
		Score (Logrank) Test & 22.530$^{***}$ (df = 2) \\ 
		\hline 
		\hline \\[-1.8ex] 
		\textit{Note:}  & \multicolumn{1}{r}{$^{*}$p$<$0.1; $^{**}$p$<$0.05; $^{***}$p$<$0.01} \\ 
	\end{tabular} 
\end{table} 

m.age: The coefficient is 0.008 with a standard error of 0.002. This suggests that for each unit increase in mother's age, the entry time increases by 0.008 units. The coefficient is statistically significant at the 1% level (***).
sexfemale: The coefficient is -0.082 with a standard error of 0.027. This suggests that being female (compared to male) decreases the entry time by 0.082 units. The coefficient is statistically significant at the 1% level (***).



\vspace{0.5cm}
\end{document}
